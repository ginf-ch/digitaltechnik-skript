% !TEX root = ../../../main.tex

\toggletrue{image}
\toggletrue{imagehover}
\chapterimage{1_to_10}
\chapterimagetitle{1 TO 10}
\chapterimageurl{https://xkcd.com/953/}
\chapterimagehover{If you get an 11/100 on a CS test, but you claim it should be counted as a \say{C}, they'll probably decide you deserve the upgrade.}

\chapter{Dualcode}
\label{ch:dualcode}

In diesem Kapitel geht es darum, wie wir eine natürliche Dezimalzahl, also z.B. \num{42}, durch Nullen und Einsen darstellen. Die Lernziele dieses Kapitels sind:\\

\lernziel[\autoref{part:dualsystem}, Dualsystem]{\autoref{ch:dualcode}, \nameref{ch:dualcode}}{
\begin{minipage}{\linewidth}
$\square$ \hspace{0.1cm} Sie definieren die Begriffe Binärziffer, Binärzahl, Bit und Dualzahl.\\
$\square$ \hspace{0.1cm} Sie wandeln eine gegebene Dezimalzahl in die entsprechende Dualzahl um.\\
$\square$ \hspace{0.1cm} Sie wandeln eine gegebene Dualzahl in die entsprechende Dezimalzahl um.\\
$\square$ \hspace{0.1cm} Sie zählen mithilfe von Dualzahlen, ohne umzuwandeln.\\
$\square$ \hspace{0.1cm} Sie erklären anhand eines Beispiels, wie der Dualcode aufgebaut ist.
\end{minipage}
}

\section{Grundbegriffe}

Der \textbf{Dualcode} wandelt eine natürliche Dezimalzahl in eine binäre Ziffernfolge um und umgekehrt.

\begin{definition}[Binärziffer]
Die Ziffern $0$ und $1$ werden Binärziffern (eng. binary digits) genannt.
\end{definition}

Noch einmal zur Verdeutlichung: Es gibt genau zwei unterschiedliche Binärziffern. $0$ ist eine Binärziffer und $1$ ist eine Binärziffer\footnote{Sie kennen den Begriff einer Ziffer bereits aus dem Dezimalsystem. Dort gibt es $10$ verschiedene Dezimalziffern: $0$, $1$, $2$, $3$, $4$, \dots, $9$.}. Nun können wir den Begriff einer Binärzahl definieren.

\begin{definition}[Binärzahl]
Eine Binärzahl ist eine beliebige Folge von Binärziffern.
\end{definition}

\begin{example}
$0011001110011$ und $111000$ sind Beispiele für Binärzahlen.
\end{example}

\begin{definition}[Bit]
Eine einzelne Stelle einer Binärzahl wird \textbf{Bit} genannt. Dies ist ein Synonym für Binärstelle. Die Stellen einer Binärzahl werden von \textbf{rechts} nach \textbf{links} nummeriert, beginnend mit $0$ für die \textbf{rechteste Stelle}. Besitzt eine Binärzahl $n$-Stellen, dann sprechen wir auch von einer $n$-Bit-Binärzahl oder kurz von einer $n$-Bit-Zahl.
\end{definition}

\begin{example}
Die Binärzahl $1101$ ist eine $4$-Bit-Binärzahl. Sie hat vier Binärstellen, das heisst \textbf{vier Bits}. Die Nummerierung der Bits ist in der folgenden Tabelle dargestellt.

\begin{table}[H]
\centering
\begin{tabular}{|c|c|c|c|}
\hline
$1$         & $1$         & $0$         & $1$         \\ \hline
3. Stelle & 2. Stelle & 1. Stelle & 0. Stelle \\ \hline
\end{tabular}
\end{table}
\end{example}

\begin{definition}[Dualzahlen]
Wir können eine Binärzahl als die Darstellung einer natürlichen Dezimalzahl interpretieren. Wir sprechen dann von einer Dualzahl und nicht von einer Binärzahl.
\end{definition}

\begin{example}
Interpretieren wir $111001$ als die Darstellung einer natürlichen Dezimalzahl, dann ergibt sich die Dezimalzahl $57$. Wir bezeichnen $111001$ als Dualzahl. Die Dualzahl $101$ besteht aus $3$ Binärziffern und entspricht der natürlichen Dezimalzahl $5$.
\end{example}

\section{Dezimalzahl $\Rightarrow$ Dualzahl}

Zuerst betrachten wir, wie wir systematisch eine beliebige, natürliche Dezimalzahl in eine Dualzahl umwandeln können.

\begin{example}
Wir wandeln die Zahl $93_{10}$ in eine Dualzahl um.\\

\begin{minipage}[c][4cm]{0.3\linewidth}
\begin{alignat*}{6}
93 & : & ~2 & ~=~ & 46 & ~R~ & 1 \tikzmark{firstDigit} \\
46 & : & ~2 & ~=~ & 23 & ~R~ & 0 \\
23 & : & ~2 & ~=~ & 11 & ~R~ & 1 \\
11 & : & ~2 & ~=~ & 5 & ~R~ & 1 \\
5 & : & ~2 & ~=~ & 2 & ~R~ & 1 \\
2 & : & ~2 & ~=~ & 1 & ~R~ & 0 \\
1 & : & ~2 & ~=~ & 0 & ~R~ & 1 \tikzmark{lastDigit} \\
\end{alignat*}
\end{minipage}
\hfill
\begin{minipage}[c][4cm]{0.6\linewidth}
\textbf{Rechenweg:}
\begin{enumerate}
\item Dividieren mit Rest: Dezimalzahl durch $2$ teilen und den Rest notieren.
\item Die Division mit dem Ergebnis wiederholen, bis das Ergebnis $0$ ist.
\item Das Ergebnis ablesen. Die Binärziffern der Dualzahl (von links nach rechts) sind die Reste, die von unten nach oben gelesen werden.
\end{enumerate}
\end{minipage}

\tikz[remember picture] \draw[overlay, ->] ([xshift = 0.5cm]pic cs:lastDigit) -- ([xshift = 0.5cm]pic cs:firstDigit);

Wir erhalten die Dualzahl, indem wir die \say{Reste} (Binärziffern) von \say{unten} nach \say{oben} aufschreiben. Im Beispiel ist das $1011101_2 = 93_{10}$.

\end{example}

\subsection{Was bedeuten die kleinen Zahlen?}

Damit wir wissen, wie eine Zahl zu \say{lesen} ist, notieren wir hinter der Zahl ein Subskript\footnote{tiefgestelltes Druckzeichen}. Für Dezimalzahlen verwenden wir das Subskript $_{10}$, für Dualzahlen $_2$.

\begin{example}
$19_{10} = \underbrace{1001\overbrace{1}^{\textrm{Binärziffer}}}_{\textrm{Dualzahl}}$$~_2$
\end{example}

\section{Dualzahl $\Rightarrow$ Dezimalzahl}

Nun geht es darum, wie wir systematisch eine Dualzahl in eine Dezimalzahl umwandeln. Die Vorgehensweise ist wie folgt: Schreibe unter die Binärziffern die $2$-er Potenzen. Beginne mit der Binärziffer ganz rechts und notiere darunter die Zahl $1$. Verdopple pro Binärziffer die Zahl. Addiere nur diejenigen $2$-er Potenzen, welche unterhalb einer $1$ stehen.

\begin{example}
Wir wandeln die Zahl $1011101_{2}$ in eine Dezimalzahl um.

\begin{table}[htb]
\centering
\begin{tblr}{|r|r|r|r|r|r|r|}
\hline
1  & 0  & 1  & 1 & 1 & 0 & 1 \\ \hline
64 & 32 & 16 & 8 & 4 & 2 & 1 \\ \hline[2pt]
64 & 0  & 16 & 8 & 4 & 0 & 1 \\ \hline
\end{tblr}
\end{table}

$\Rightarrow 64 + 16 + 8 + 4 + 1 = 93_{10} = 1011101_{2}$

\end{example}

\section{Formeln im Umgang mit dem Dualcode}
\label{sec:formeln-dualcode}
Handelt es sich bei der Dezimalzahl um eine $2$-er Potenz, dann können wir schneller umwandeln:

\begin{center}
$(2^k)_{10}=1\overbrace{0\dots0}^{k~\textrm{Nullen}}$$_2$
\end{center}

\begin{example}
\label{ex:analyse-64}
Wir möchten die Dezimalzahl $64$ codieren. Wir erhalten: $64_{10} = 2^6_{10} = 1000000_2$.

\end{example}

Oft interessiert uns, wie viele \textbf{verschiedene} Dualzahlen wir mit einer bestimmten Anzahl von Binärziffern darstellen können. Wir berechnen dies wie folgt: Mit $n$ Binärziffern können wir $2^n$ verschiedene Dualzahlen erzeugen. Die kleinste Dualzahl ist $0_2$ und die grösste Dualzahl ist $\underbrace{1\dots1}_{n~\textrm{Einsen}}$$_2$.

\begin{example}
Wie viele (verschiedene) Dualzahlen können wir mit $4$ Binärziffern darstellen? Zuerst berechnen wir die Anzahl der Dualzahlen: $2^4 = 16$. Dann bestimmen wir die kleinste ($0_2 = 0_{10}$) und die grösste Dualzahl ($1111_2 = 15_{10}$).
\end{example}

\section{Wie wir mit Dualzahlen zählen}

\autoref{table-counting-numbers} zeigt die ersten $32$ natürlichen Dezimalzahlen. Daneben steht jeweils die entsprechende Dualzahl. Wir erkennen bei den Dualzahlen ein Muster. Besteht eine Dualzahl nur aus Einsen, so hat die nächste Dualzahl eine Stelle mehr und besteht nur aus einer Eins und ansonsten aus Nullen. Ausserdem wechseln sich die Binärziffern an der ersten Stelle (ganz rechts) immer ab (erst $0$, dann $1$, dann wieder $0$, dann wieder $1$ $\dots$). Beim Zählen mit Dualzahlen werden alle Kombinationen von Einsen und Nullen systematisch aufgelistet.

\begin{table}[htb]
\centering
\begin{tabular}{rrc||rrc}
\toprule
\small{\textbf{Dezimalzahl}} &  \small{\textbf{Dualzahl}} & \small{\textbf{Kommentar}} &  \small{\textbf{Dezimalzahl}} &  \small{\textbf{Dualzahl}} &  \small{\textbf{Kommentar}} \\
\midrule
0 & 0 & - & 16 & 10000 & $2^4$ \\
1 & 1 & - & 17 & 10001 & - \\
2 & 10 & - & 18 & 10010 & - \\
3 & 11 & $2^2-1$ & 19 & 10011 & - \\
4 & 100 & $2^2$ & 20 & 10100 & - \\
5 & 101 & - & 21 & 10101 & - \\
6 & 110 & - & 22 & 10110 & - \\
7 & 111 & $2^3-1$ & 23 & 10111 & - \\
8 & 1000 & $2^3$ & 24 & 11000 & - \\
9 & 1001 & - & 25 & 11001 & - \\
10 & 1010 & - & 26 & 11010 & - \\
11 & 1011 & - & 27 & 11011 & - \\
12 & 1100 & - & 28 & 11100 & - \\
13 & 1101 & - & 29 & 11101 & - \\
14 & 1110 & - & 30 & 11110 & - \\
15 & 1111 & $2^4-1$ & 31 & 11111 & $2^5-1$ \\
\bottomrule
\end{tabular}
\caption{Die ersten $32$ Dezimalzahlen werden als Dualzahlen dargestellt.}
\label{table-counting-numbers}
\end{table}

Dualzahlen benötigen mehr Stellen als die entsprechenden Dezimalzahlen (mit der Ausnahme von $0$ und $1$). Der Grund dafür ist, dass für Dualzahlen nur zwei Ziffern zur Verfügung stehen, für Dezimalzahlen jedoch zehn.

\section{Wie ist der Dualcode aufgebaut?}
\label{sec:aufbau-dualcode}

Der Dualcode ist eigentlich nichts anderes als eine mathematische Umrechnung. Dabei wird eine Zahl aus dem Dezimalsystem (Dezimalzahl) einfach in die entsprechende Zahl aus dem Dualsystem (Dualzahl) umgewandelt. In diesem Abschnitt betrachten wir den Aufbau des Dualsystems, das die Grundlage für den Dualcode bildet. Dualzahlen und Dezimalzahlen haben den gleichen Aufbau.

\begin{example}
\label{ex:analyse-31}
Wir analysieren die Zahl $31$.

\begin{minipage}{0.6\textwidth}
\begin{alignat*}{16}
31 & ~ = ~ & & & 16 & ~ + ~ & & & 8 & ~ + ~ & & & 4 & ~ + ~ & & & 2 & ~ + ~ & & & 1 \\
& ~ = ~ & 1 & ~\cdot~ & 16 & ~ + ~ & 1 & ~\cdot~ & 8 & ~ + ~ & 1 & ~\cdot~ & 4 & ~ + ~ & 1 & ~\cdot~ & 2 & ~ + ~ & 1 & ~\cdot~ & 1 \\
& ~ = ~ & 1 & ~\cdot~ & 2 \cdot 2 \cdot 2 \cdot 2  & ~ + ~ & 1 & ~\cdot~ & 2 \cdot 2 \cdot 2  & ~ + ~ & 1 & ~\cdot~ & 2 \cdot 2 & ~ + ~ & 1 & ~\cdot~ & 2 & ~ + ~ & 1 & ~\cdot~ & 1 \\
& ~ = ~ & 1 & ~\cdot~ & \underset{\Uparrow}{2}^4  & ~ + ~ & 1 & ~\cdot~ & \underset{\Uparrow}{2}^3  & ~ + ~ & 1 & ~\cdot~ & \underset{\Uparrow}{2}^2 & ~ + ~ & 1 & ~\cdot~ & \underset{\Uparrow}{2}^1 & ~ + ~ & 1 & ~\cdot~ & \underset{\Uparrow}{2}^0 \\
\omit\rlap{\framebox[9cm]{Basis: $2$ $\Rightarrow$ \textbf{Dual}system\footnotemark}}
\end{alignat*}
\end{minipage}
\hfill
\begin{minipage}{0.35\textwidth}
\begin{alignat*}{6}
2^0 & ~ = ~ & 1 & ~~~~ 	2^6 & ~ = ~ & 64  \\
2^1 & ~ = ~ & 2	 & ~~~~	2^7 & ~ = ~ & 128  \\
2^2 & ~ = ~ & 4	 & ~~~~	2^8 & ~ = ~ & 256  \\
2^3 & ~ = ~ & 8	 & ~~~~	2^9 & ~ = ~ & 512  \\
2^4 & ~ = ~ & 16 &~~~~ 	2^{10} & ~ = ~ & 1024  \\
2^5 & ~ = ~ & 32 &~~~~ 	2^{11} & ~ = ~ & 2048  \\
\end{alignat*}
\end{minipage}
Die Zahl $31$ lässt sich durch Potenzen zur Basis $2$ darstellen. Jede Potenz wird mit der Zahl $1$ multipliziert.  Wenn wir nun nur die ersten Faktoren der Multiplikation notieren, dann erhalten wir $11111$. Dies ist eine Zahl aus dem Dualsystem. Es ist die gleiche Zahl, die wir erhalten würden, wenn wir das Kochrezept zum Umwandeln anwenden würden.

\end{example}

\footnotetext{Auch Zweiersystem genannt.}

Auch die Position einer Ziffer im Dualsystem ist wichtig. Jede Stelle hat einen Stellenwert. Die Stelle mit dem niedrigsten Stellenwert steht ganz \textbf{rechts}. Die erste Stelle (ganz rechts) hat den Wert $1$, die zweite Stelle hat den Wert $2$, die dritte Stelle hat den Wert $4$, die vierte Stelle hat den Wert $8$, die fünfte Stelle hat den Wert $16$ usw. Es sind gerade die $2$-er Potenzen: $2^0$, $2^1$, $2^2$, $2^3$ und $2^4$.

\begin{example}
\label{ex:analyse-42}
\begin{alignat*}{16}
42 & ~ = ~ & & & 32 & ~ + ~ & & & & & & & 8 & ~ + ~ & & & & & & & 2 &\\
& ~ = ~ & 1 & ~\cdot~ & 32 & ~ + ~ & 0 & ~\cdot~ & 16 & ~ + ~ & 1 & ~\cdot~ & 8 & ~ + ~ & 0 & ~\cdot~ & 4 & ~ + ~ & 1 & ~\cdot~ & 2 & ~ + ~ & 0 & ~\cdot~ & 1 \\
& ~ = ~ & \underset{\Uparrow}{1} & ~\cdot~ & 2^5  & ~ + ~ & \underset{\Uparrow}{0} & ~\cdot~ & 2^4  & ~ + ~ & \underset{\Uparrow}{1} & ~\cdot~ & 2^3 & ~ + ~ & \underset{\Uparrow}{0} & ~\cdot~ & 2^2 & ~ + ~ & \underset{\Uparrow}{1} & ~\cdot~ & 2^1 & ~ + ~ & \underset{\Uparrow}{0} & ~\cdot~ & 2^0 \\
\omit\rlap{\framebox[10cm]{Ziffern des Dualsystems: $0$ und $1$}}
\end{alignat*}

Im Dualsystem gibt es nur die Ziffern $0$ und $1$. Wir können die Dezimalzahl $42$ also auch durch die Duahlzahl $101010$ darstellen. Wir schreiben $42_{10} = 101010_2$.

\end{example}

Wir können eine natürliche Dezimalzahl auch durch eine Zerlegung der Dezimalzahl in eine Summe von $2$-er Potenzen in eine Dualzahl umwandeln. Dazu müssen wir jedoch sattelfest bei den $2$-er Potenzen sein und die grösste $2$-er Potenz finden, die gerade noch in die Dezimalzahl \say{passt}. Das ist nicht immer ganz einfach. Insbesondere bei grösseren Zahlen.

\begin{example}
\label{ex:analyse-2048}
\begin{align*}
2048 & ~ = ~ & 1 & ~\cdot~ & 2048  & ~ + ~ & 0 & ~\cdot~ & 1024 & ~ + ~ & 0 & ~\cdot~ & 512 & ~ + ~ & 0 & ~\cdot~ & 256 & ~ + ~ & 0 & ~\cdot~ & 128 \\ 
& ~ + ~ & 0 & ~\cdot~ & 64 & ~ + ~ & 0 & ~\cdot~ & 32 & ~ + ~ & 0 & ~\cdot~ & 16 & ~ + ~ & 0 & ~\cdot~ & 8 & ~ + ~ & 0 & ~\cdot~ & 4 \\
& ~ + ~ & 0 & ~\cdot~ & 2 & ~ + ~ & 0 & ~\cdot~ & 1  \\
& ~ = ~ & 1 & ~\cdot~ & 2^{11}  & ~ + ~ & 0 & ~\cdot~ & 2^{10}  & ~ + ~ & 0 & ~\cdot~ & 2^9 & ~ + ~ & 0 & ~\cdot~ & 2^8 & ~ + ~ & 0 & ~\cdot~ & 2^7 \\ 
& ~ + ~ & 0 & ~\cdot~ & 2^6 & ~ + ~ & 0 & ~\cdot~ & 2^5 & ~ + ~ & 0 & ~\cdot~ & 2^4 & ~ + ~ & 0 & ~\cdot~ & 2^3 & ~ + ~ & 0 & ~\cdot~ & 2^2 \\ 
& ~ + ~ & 0 & ~\cdot~ & 2^1 & ~ + ~ & 0 & ~\cdot~ & 2^0
\end{align*}

Wir erhalten $2048_{10} = 100000000000_2$.

\end{example}

\newpage

\section{Übungen}

\begin{exercise}
Wandeln Sie die folgenden Dezimalzahlen in die entsprechenden Dualzahlen um. Stellen Sie die Rechenschritte sauber dar.

\begin{multicols}{2}
\begin{enumerate}

\item $67_{10} = $ \\
\begin{minipage}{\linewidth}
\centering
\fillwithgrid{2.25in}
\end{minipage}

\item $1000_{10} = $ \\
\begin{minipage}{\linewidth}
\centering
\fillwithgrid{2.75in}
\end{minipage}

\item $333_{10} = $ \\
\begin{minipage}{\linewidth}
\centering
\fillwithgrid{2.75in}
\end{minipage}

\item $37_{10} = $ \\
\begin{minipage}{\linewidth}
\centering
\fillwithgrid{2.25in}
\end{minipage}

\item $851_{10} = $ \\
\begin{minipage}{\linewidth}
\centering
\fillwithgrid{2.75in}
\end{minipage}

\item $442_{10} = $ \\
\begin{minipage}{\linewidth}
\centering
\fillwithgrid{2.75in}
\end{minipage}

\end{enumerate}
\end{multicols}
\end{exercise}

\begin{exercise}
Wandeln Sie die folgenden Dualzahlen in die entsprechenden Dezimalzahlen um. Stellen Sie die Rechenschritte sauber dar.

\begin{multicols}{2}
\begin{enumerate}

\item $1111_2= $ \\
\begin{minipage}{\linewidth}
\centering
\fillwithgrid{1in}
\end{minipage}

\item $100000_2= $ \\
\begin{minipage}{\linewidth}
\centering
\fillwithgrid{1in}
\end{minipage}

\item $101101_2= $ \\
\begin{minipage}{\linewidth}
\centering
\fillwithgrid{1in}
\end{minipage}

\item $1001_2= $ \\
\begin{minipage}{\linewidth}
\centering
\fillwithgrid{1in}
\end{minipage}

\item $110101_2= $ \\
\begin{minipage}{\linewidth}
\centering
\fillwithgrid{1in}
\end{minipage}

\item $111000_2= $ \\
\begin{minipage}{\linewidth}
\centering
\fillwithgrid{1in}
\end{minipage}

\end{enumerate}
\end{multicols}
\end{exercise}

\begin{exercise}
\begin{enumerate}
\item Notieren Sie eine Binärzahl mit $3$ Bit.
\fillwithlines{0.25in}
\item Wie viele Bits benötigt die Dualzahl, die der Dezimalzahl $67$ entspricht?
\fillwithlines{0.25in}
\item Es ist die Dualzahl $10101110$ gegeben.
\begin{multicols}{2}
\begin{enumerate}
\item Wie lautet das $0.$ Bit?
\fillwithlines{0.25in}
\item Wie lautet das $3.$ Bit?
\fillwithlines{0.25in}
\item Wie lautet das $7.$ Bit?
\fillwithlines{0.25in}
\item Wie wie viele Bits besitzt die Zahl?
\fillwithlines{0.25in}
\end{enumerate}
\end{multicols}
\end{enumerate}
\end{exercise}

\begin{exercise}
Lesen Sie den Anfang von \autoref{sec:formeln-dualcode} bis und mit \autoref{ex:analyse-64} durch. Finden Sie dann eine \textbf{Formel} für folgende Frage: Wie viele \textbf{Bits} werden für eine \textbf{Dezimalzahl} benötigt, wenn die Dezimalzahl eine \textbf{Zweierpotenz} darstellt. Überprüfen Sie Ihre Formel mit mindestens \textbf{drei} sinnvollen Beispielen.
\fillwithgrid{\stretch{1}}
\end{exercise}

\begin{exercise}
Wie viele Dualzahlen können mit der folgenden Anzahl von Binärziffern dargestellt werden? Geben Sie jeweils die grösste Dualzahl und die entsprechende Dezimalzahl an.

\begin{enumerate}
\item $4$ Binärziffern
\fillwithlines{0.25in}
\item $8$ Binärziffern
\fillwithlines{0.25in}
\item $12$ Binärziffern
\fillwithlines{0.25in}
\end{enumerate}
\end{exercise}

\vspace{-0.35cm}

\begin{exercise}
Wie viele Dualzahlen können mit der folgenden Anzahl von Bits dargestellt werden? Geben Sie jeweils die grösste Dualzahl und die entsprechende Dezimalzahl an.

\begin{enumerate}
\item $5$ Bits
\fillwithlines{0.25in}
\item $10$ Bits
\fillwithlines{0.25in}
\item $16$ Bits
\fillwithlines{0.25in}
\end{enumerate}
\end{exercise}

\vspace{-0.35cm}

\begin{exercise}
\begin{enumerate}
\item Wie können Sie bei einer Dualzahl einfach feststellen, ob es sich um eine gerade/ungerade Zahl im Dezimalsystem handelt?
\fillwithgrid{0.5in}
\item Geben Sie eine Formel an, wie wir für eine Dualzahl mit $n$ Einsen ($\overbrace{111\dots1}^{n-\textrm{Einsen}}$~$_2$) die entsprechende Dezimalzahl berechnen können. Die Formel soll möglichst kompakt sein.
\fillwithgrid{0.5in}
\end{enumerate}
\end{exercise}

\vspace{-0.35cm}

\begin{exercise}
Betrachten Sie \autoref{ex:analyse-31}, \autoref{ex:analyse-42} und \autoref{ex:analyse-2048} in \autoref{sec:aufbau-dualcode}. Nehmen Sie dann Ihr \textbf{Geburtsjahr} (Format: YYYY) und stellen Sie es so wie in den Beispielen dar. Sie müssen Ihr Geburtsjahr also in eine Summe von Zweierpotenzen zerlegen. Markieren Sie die \textbf{Basis} und \textbf{Ziffern} eindeutig und mit den korrekten \textbf{Fachbegriffen}. Leiten Sie anschliessend die \textbf{Dualzahl} ab. Wie viele \textbf{Bits} werden benötigt?
\fillwithgrid{\stretch{1}}
\end{exercise}

\begin{exercise}
\begin{enumerate}
\item Die Dualzahl $10111011$ ist gegeben. Listen Sie die nächsten $10$ Dualzahlen auf.
\fillwithgrid{1in}
\item Die Dualzahl $111111$ ist gegeben. Listen Sie die nächsten $10$ Dualzahlen auf.
\fillwithgrid{1in}
\item Die Dualzahl $10101110$ ist gegeben. Listen Sie die nächsten $10$ Dualzahlen auf.
\fillwithgrid{1in}
\item Die Dualzahl $10011011$ ist gegeben. Listen Sie die nächsten $10$ Dualzahlen auf.
\fillwithgrid{1in}
\item Die Dualzahl $111000$ ist gegeben. Listen Sie die nächsten $10$ Dualzahlen auf.
\fillwithgrid{1in}
\end{enumerate}
\end{exercise}
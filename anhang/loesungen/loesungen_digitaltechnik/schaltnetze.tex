\section{Lösungsvorschlag: \autoref{chapter-digitaltechnik}, \nameref{chapter-digitaltechnik}}

\subsection*{\ref{subsection-aufgaben-digitaltechnik-schaltnetze} \nameref{subsection-aufgaben-digitaltechnik-schaltnetze}}

\begin{enumerate}
	\item $A_1 = E_1 \vee \neg E_2 \Rightarrow A_1 = (E_1 \vee (\neg E_2))$

\begin{figure}[htb]
\centering
\begin{minipage}{0.45\textwidth}
\centering
\begin{circuitikz}
\draw (0,0) node[buffer port] (A) {}
(A.in 1) node[anchor=east] {$E_2$}; 
\node at (A.bout) [ocirc, right]{};
\draw (A.out) -- ++(0.5,0) node[or port, anchor=in 2] (B) {};
\draw (A.in 1) ++ (0, 1) node[anchor=east] {$E_1$} -- (0.5,1) |- (B.in 1);
\draw (B.out) node[anchor=west] {$A_1$};
\end{circuitikz}
\caption{Schaltnetz}
\label{figure-digitaltechnik-aufgabe1-schaltnetz}
\end{minipage}
\hfill
\begin{minipage}{0.45\textwidth}
\centering
\begin{tabular}{|c|c||c|}
\hline
$E_1$ 	& 	$E_2$ 	& 	$A_1 = (E_1 \vee (\neg E_2))$ 	\\ \hline
$0$		&  	$0$     	& 	$1$    					\\ \hline
$0$		& 	$1$     	& 	$0$   					\\ \hline
$1$ 	& 	$0$      	& 	$1$   					\\ \hline
$1$		& 	$1$     	& 	$1$     					\\ \hline
\end{tabular}
\caption{Wahrheitstabelle}
\label{table-digitaltechnik-aufgabe1-ttt}
\end{minipage}
\end{figure}

\item $A_1 = \neg (E_1 \wedge E_2) \Rightarrow A_1 = (\neg (E_1 \wedge E_2))$

\begin{figure}[htb]
\centering
\begin{minipage}{0.45\textwidth}
\centering
\begin{circuitikz}
\draw node[and port] (A) {} 
(A.in 1) node[anchor=east] {$E_1$}
(A.in 2) node[anchor=east] {$E_2$}
(A.out) node[buffer port, anchor=west] (B) {}
(B.in 1) node[anchor=east] {}; 
\node at (B.bout) [ocirc, right]{};
\draw (B.out) node[anchor=west] {$A_1$};
\end{circuitikz}
\caption*{Schaltnetz}
\label{figure-digitaltechnik-aufgabe2-schaltnetz}
\end{minipage}
\hfill
\begin{minipage}{0.45\textwidth}
\centering
\begin{tabular}{|c|c||c|}
\hline
$E_1$ 	& 	$E_2$ 	& 	$A_1 = (\neg (E_1 \wedge E_2))$ 	\\ \hline
$0$		&  	$0$     	& 	$1$    						\\ \hline
$0$		& 	$1$     	& 	$1$   						\\ \hline
$1$ 		& 	$0$      	& 	$1$   						\\ \hline
$1$		& 	$1$     	& 	$0$     						\\ \hline
\end{tabular}
\caption{Wahrheitstabelle}
\label{figure-digitaltechnik-aufgabe2-ttt}
\end{minipage}
\end{figure}

Das Schaltnetz für die Boolesche Formel $A_1 = \neg (E_1 \wedge E_2)$ kommt sehr häufig vor. Deshalb kann man das Schaltnetz direkt als eigenes Logikgatter kaufen. Es ist unter dem Namen \textbf{\texttt{NAND}} (von engl. \textbf{n}ot \textbf{and}) bekannt und ist in \autoref{figure-circuit-nand} dargestellt.

\begin{figure}[htb]
\centering
\begin{circuitikz}
\draw (0,0) node[and port] (myand) {}
(myand.in 1) node[anchor=east] {$E_1$} 
(myand.in 2) node[anchor=east] {$E_2$}
(myand.out) node[anchor=west] {$A_1$};
\node at (myand.bout) [ocirc, right]{};
\end{circuitikz}
\caption{Das \texttt{NAND}-Gate besteht aus einem \texttt{AND}-Gate mit einem Kreis hinter dem Baustein. Der Kreis entspricht dem \texttt{NOT}-Gate.}
\label{figure-circuit-nand}
\end{figure}

\newpage

\item $A_1 = \neg (E_1 \vee E_2) \Rightarrow A_1 = (\neg (E_1 \vee E_2))$

\begin{figure}[htb]
\centering
\begin{minipage}{0.45\textwidth}
\centering
\begin{circuitikz}
\draw node[or port] (A) {} 
(A.in 1) node[anchor=east] {$E_1$}
(A.in 2) node[anchor=east] {$E_2$}
(A.out) node[buffer port, anchor=west] (B) {}
(B.in 1) node[anchor=east] {}; 
\node at (B.bout) [ocirc, right]{};
\draw (B.out) node[anchor=west] {$A_1$};
\end{circuitikz}
\caption{Schaltnetz}
\label{figure-digitaltechnik-aufgabe3-schaltnetz}
\end{minipage}
\hfill
\begin{minipage}{0.45\textwidth}
\centering
\begin{tabular}{|c|c||c|}
\hline
$E_1$ 	& 	$E_2$ 	& 	$A_1 = (\neg (E_1 \vee E_2))$ 	\\ \hline
$0$		&  	$0$     	& 	$1$    					\\ \hline
$0$		& 	$1$     	& 	$0$   					\\ \hline
$1$ 		& 	$0$      	& 	$0$   					\\ \hline
$1$		& 	$1$     	& 	$0$     					\\ \hline
\end{tabular}
\caption{Wahrheitstabelle}
\label{figure-digitaltechnik-aufgabe3-ttt}
\end{minipage}
\end{figure}

Das Schaltnetz für $A_1$ kommt häufig vor. Deshalb kann man es als Logikgatter kaufen. Es ist unter dem Namen \textbf{\texttt{NOR}} (von engl. \textbf{n}ot \textbf{or}) bekannt und ist in \autoref{figure-circuit-nor} dargestellt.

\begin{figure}[htb]
\centering
\begin{circuitikz}
\draw (0,0) node[or port] (myor) {}
(myor.in 1) node[anchor=east] {$E_1$} 
(myor.in 2) node[anchor=east] {$E_2$}
(myor.out) node[anchor=west] {$A_1$};
\node at (myor.bout) [ocirc, right]{};
\end{circuitikz}
\caption{Das \texttt{NOR}-Gate besteht aus einem \texttt{OR}-Gate mit einem Kreis hinter dem Baustein.}
\label{figure-circuit-nor}
\end{figure}

\item $A_1 = E_1 \wedge \neg E_2 \vee \neg E_1 \wedge E_2 \Rightarrow A_1 = ((E_1 \wedge (\neg E_2)) \vee ((\neg E_1) \wedge E_2))$

\begin{figure}[htb]
\centering
\begin{minipage}{0.45\textwidth}
\centering
\begin{circuitikz}
\draw (-2, 1) node[and port] (A) {};
\draw (-2, -1) node[and port] (B) {}; 
\draw (0,0) node[or port] (C) {}; 
\draw (A.out) |- (C.in 1);
\draw (B.out) |- (C.in 2);
\draw (C.out) node[anchor=west] {$A_1$};
\draw (-4,-0.5) node[buffer port] (D) {};
\node at (D.bout) [ocirc, right]{};
\draw (-4,0.5) node[buffer port] (E) {};
\node at (E.bout) [ocirc, right]{};
\draw (D.out) |- (B.in 1);
\draw (E.out) |- (A.in 2);
\draw (-7, 0.5) node (E1) {$E_1$};
\draw (E1) (-6.25,  0.5) node[circle,fill,inner sep=1pt] (E1cross) {} |- (D.in 1);
\draw (E1) to (-6.25, 0.5) |- (A.in 1);
\draw (-7, -0.75) node (E2) {$E_2$};
\draw (E2) (-6,  -0.75) node[circle,fill,inner sep=1pt] (E2cross) {} |- (E.in 1);
\draw (E2) to (-6, -0.75) |- (B.in 2);
\end{circuitikz}
\caption{Schaltnetz}
\label{figure-digitaltechnik-aufgabe4-schaltnetz}
\end{minipage}
\hfill
\begin{minipage}{0.45\textwidth}
\centering
\begin{tabular}{|c|c||c|}
\hline
$E_1$ 	& 	$E_2$ 	& 	$A_1 = ((E_1 \wedge (\neg E_2)) \vee ((\neg E_1) \wedge E_2))$ 	\\ \hline
$0$		&  	$0$     	& 	$0$    												\\ \hline
$0$		& 	$1$     	& 	$1$   												\\ \hline
$1$ 		& 	$0$      	& 	$1$   												\\ \hline
$1$		& 	$1$     	& 	$0$     												\\ \hline
\end{tabular}
\caption{Wahrheitstabelle}
\label{figure-digitaltechnik-aufgabe4-ttt}
\end{minipage}
\end{figure}

Das Schaltnetz für $A_1$ kommt häufig vor. Es ist unter dem Namen \textbf{Antivalenz} oder \textbf{\texttt{XOR}} (engl. e\textbf{x}clusive \textbf{or}) bekannt und ist in \autoref{figure-circuit-xor} dargestellt.

\begin{figure}[H]
\centering
\begin{circuitikz}
\draw (0,0) node[xor port] (myxor) {}
(myxor.in 1) node[anchor=east] {$E_1$} 
(myxor.in 2) node[anchor=east] {$E_2$}
(myxor.out) node[anchor=west] {$A_1$};
\end{circuitikz}
	\caption{Das \texttt{XOR}-Gate besteht aus einem Gleichheitszeichen mit einer 1 im Rechteck.}
\label{figure-circuit-xor}	
\end{figure}

\item $A_1 = E_1 \wedge E_2 \vee \neg E_1 \wedge \neg E_2 \Rightarrow A_1 = ((E_1 \wedge E_2) \vee ((\neg E_1) \wedge (\neg E_2)))$

\begin{figure}[H]
\centering
\begin{minipage}{0.45\textwidth}
\centering
\begin{circuitikz}
\draw (-2, 1) node[and port] (A) {};
\draw (-2, -1) node[and port] (B) {}; 
\draw (0,0) node[or port] (C) {}; 
\draw (A.out) |- (C.in 1);
\draw (B.out) |- (C.in 2);
\draw (C.out) node[anchor=west] {$A_1$};
\draw (-4,-0.5) node[buffer port] (D) {};
\node at (D.bout) [ocirc, right]{};
\draw (-4,-1.5) node[buffer port] (E) {};
\node at (E.bout) [ocirc, right]{};
\draw (D.out) |- (B.in 1);
\draw (E.out) |- (B.in 2);
\draw (-7, 0.5) node (E1) {$E_1$};
\draw (E1) (-6.25,  0.5) node[circle,fill,inner sep=1pt] (E1cross) {} |- (D.in 1);
\draw (E1) to (-6.25, 0.5) |- (A.in 1);
\draw (-7, -0.75) node (E2) {$E_2$};
\draw (E2) (-6,  -0.75) node[circle,fill,inner sep=1pt] (E2cross) {} |- (E.in 1);
\draw (E2) to (-6, -0.75) |- (A.in 2);
\end{circuitikz}
\caption{Schaltnetz}
\label{figure-digitaltechnik-aufgabe5-schaltnetz}
\end{minipage}
\hfill
\begin{minipage}{0.45\textwidth}
\centering
\begin{tabular}{|c|c||c|}
\hline
$E_1$ 	& 	$E_2$ 	& 	$A_1 = ((E_1 \wedge E_2) \vee ((\neg E_1) \wedge (\neg E_2)))$ 	\\ \hline
$0$		&  	$0$     	& 	$1$    												\\ \hline
$0$		& 	$1$     	& 	$0$   												\\ \hline
$1$ 		& 	$0$      	& 	$0$   												\\ \hline
$1$		& 	$1$     	& 	$1$     												\\ \hline
\end{tabular}
\caption{Wahrheitstabelle}
\label{figure-digitaltechnik-aufgabe5-ttt}
\end{minipage}
\end{figure}

Das Schaltnetz für die Boolesche Formel $A_1 = ((E_1 \wedge E_2) \vee ((\neg E_1) \wedge (\neg E_2)))$ kommt sehr häufig vor. Deshalb kann man das Schaltnetz direkt als eigenes Logikgatter kaufen. Es ist unter dem Namen \textit{Äquivalenz} oder \textbf{\texttt{XNOR}} (von engl. e\textbf{x}clusive \textbf{n}ot \textbf{or}) bekannt und ist in \autoref{figure-circuit-xnor} dargestellt. Nur wenn beide Eingänge identische Werte besitzen (beide $0$ oder beide $1$), ist der Ausgang $1$\footnote{Deshalb nennt man das Logikgatter auch Äquivalenz. Beide Eingänge müssen gleichwertig (äquivalent) sein.}.

\begin{figure}[htb]
\centering
\begin{circuitikz}
\draw (0,0) node[xor port] (myxor) {}
(myxor.in 1) node[anchor=east] {$E_1$} 
(myxor.in 2) node[anchor=east] {$E_2$}
(myxor.out) node[anchor=west] {$A_1$};
\node at (myxor.bout) [ocirc, right]{};
\end{circuitikz}
\caption{Die grafische Darstellung des \texttt{XNOR}-Gates besteht aus einem \texttt{XOR}-Gate mit einem Kreis hinter dem Baustein.}
\label{figure-circuit-xnor}
\end{figure}

\end{enumerate}